\iffalse
	\title{2020}
	\author{AI24BTECH11003}
	\section{mcq-single}
\fi
 
%16
    \item If for all real triplets $\brak{a, b, c}$, $f\brak{x}=a+bx+cx^2$; then $\int_0^1f\brak{x}dx$ is equal to
    
    \hfill[Jan 2020]

        \begin{multicols}{1}
            \begin{enumerate}
                \item $2\brak{3f\brak{1}+ 2f\brak{\frac{1}{2}}}$
                \item $\brak{\frac{1}{3}}\brak{f\brak{0}+f\brak{\frac{1}{2}}}$
                \item $\brak{\frac{1}{2}}\brak{f\brak{1}+3f\brak{\frac{1}{2}}}$
                \item ${\frac{1}{6}}\brak{f\brak{0}+f\brak{1}+4f\brak{\frac{1}{2}}}$
            \end{enumerate}
        \end{multicols}

%17
    \item If the number of five digit numbers with distinct digits and 2 at the 10$^{th}$ place is $336k$, then $k$ is equal to:
    
    \hfill[Jan 2020]

		\begin{multicols}{4}
			\begin{enumerate}
				\item 8
				\item 7
				\item 4
				\item 6
			\end{enumerate}
		\end{multicols}

%18
    \item Let the observations $x_i\brak{1\leq i \leq 10}$ satisfy the equations,

    $\sum_{i=1}^{10} \brak{x_i - 5}=10$
    and
    $\sum_{i=1}^{10} \brak{x_i-5}^2=40$.
    If $\mu$ and $\lambda$ are the mean and variance of observations, $\brak{x_1-3}, \brak{x_2-3}.....\brak{x_10-3}$, then the ordered pair $\brak{\mu, \lambda}$ is equal to:
    
    \hfill[Jan 2020]

        \begin{multicols}{4}
            \begin{enumerate}
                \item $\brak{6, 3}$
                \item $\brak{3, 6}$
                \item $\brak{3, 3}$
                \item $\brak{6, 6}$
            \end{enumerate}
        \end{multicols}

%19
    \item The integral $\int\frac{dx}{\brak{x+4}^{\frac{8}{7}}\brak{x-3}^{\frac{6}{7}}}$ is equal to
    
    \hfill[Jan 2020]

		\begin{multicols}{4}
			\begin{enumerate}
				\item $-\brak{\frac{x-3}{x-4}}^{-\frac{1}{7}}+C$
				\item $\frac{1}{2}\brak{\frac{x-3}{x-4}}^{\frac{3}{7}}+C$
				\item $\brak{\frac{x-3}{x-4}}^{\frac{1}{7}}+C$
				\item $-\frac{1}{13}\brak{\frac{x-3}{x-4}}^{-\frac{13}{7}}+C$
			\end{enumerate}
		\end{multicols}

%20
    \item In a box, there are 20 cards out of which 10 are labelled as A and remaining 10 are labelled as B. Cards are drawn at random, one after the other and with replacement, till a second A-card is obtained. The probability that the second A-card appears before the third B-card is:
    
    \hfill[Jan 2020]

		\begin{multicols}{4}
			\begin{enumerate}
				\item $\frac{15}{16}$
				\item $\frac{9}{16}$
				\item $\frac{13}{16}$
				\item $\frac{11}{16}$
			\end{enumerate}
		\end{multicols}

