\iffalse
    \title{2023}
    \author{EE24BTECH11005}
    \section{integer}
\fi 
\item Let the tangents to the curve $x^2+2x-4y+9=0$ at the point $\vec{P}\myvec{1\\3}$ on it meet the y-axis at $\vec{A}$. Let the line passing through $\vec{P}$ and parallel to the line $x-3y=6$ meet the parabola $y^2=4x$ at $\vec{B}$. If $\vec{B}$ lies on the line $2x-3y=8$, then $\brak{AB}^2$ us equal to \rule{2cm}{0.2pt}\hfill{\brak{\text{Apr 2023}}}
		\item Let the point $\myvec{p\\p+1}$ lie inside the region 
			\begin{align*}
				E=\cbrak{\brak{x,y}: 3-x \le y \le \sqrt{9-x^2}, 0 \le x \le 3}
			\end{align*}
		If the set of all values of $p$ in the interval $\myvec{a\\b}$ then $b^2+b-a^2$ is equal to \rule{2cm}{0.2pt}\hfill{\brak{\text{Apr 2023}}}
	\item Let $y=y\brak{x}$ be a solution of the differential equation 
		\begin{align*}
			\brak{x\cos x}dy+\brak{xy\sin x +y\cos x-1}dx=0,0<x<\frac{\pi}{2}
		\end{align*}
		If $\frac{\pi}{3}y\brak{\frac{\pi}{3}}=\sqrt{3}$, then $\abs{\frac{\pi}{6}y^{''}\brak{\frac{\pi}{6}}+2y^{'}\brak{\frac{\pi}{6}}}$\rule{2cm}{0.2pt}\hfill{\brak{\text{Apr 2023}}}
	\item The Let $a \in Z$ and $\sbrak{t}$ be the greatest integer $\le t$. Then the number of points, where the function $f\brak{x}=\sbrak{a+13\sin x}, x\in\brak{0,\pi}$ is not differentiable is\rule{2cm}{0.2pt}\hfill{\brak{\text{Apr 2023}}}
	\item If the area of the region 
		\begin{align*}
			S=\cbrak{\brak{x,y}:2y-y^2 \le x^2 \le 2y, x \ge y}
		\end{align*}
		is equal to $\brak{\frac{n+2}{n+1}-\frac{\pi}{n-1}}$ then the natural number $n$ is equal to \rule{2cm}{0.2pt}\hfill{\brak{\text{Apr 2023}}}
	\item The number of ways of giving $20$ distinct oranges to $3$ children such that each child gets atleast one orange is \rule{2cm}{0.2pt}\hfill{\brak{\text{Apr 2023}}}
	\item Let the image of the point $\vec{P}\myvec{1\\2\\3}$ in the plane $2x-y+z=9$ be $\vec{Q}$. If the coordinates of the point $\vec{R}$ are $\myvec{6\\10\\7}$. Then the square of the area of triangle $PQR$ is \rule{2cm}{0.2pt}\hfill{\brak{\text{Apr 2023}}}
	\item Let A circle passing through the point $\vec{P}\myvec{\alpha\\\beta}$ in the first quadrant touches the two coordinate axes at the points $\vec{A},\vec{B}$. The point $\vec{P}$ is above the line $\vec{AB}$. The point $\vec{Q}$ on the line segment $\vec{AB}$ is the foot of perpendicular from $\vec{P}$ on $\vec{AB}$. If $\vec{PQ}$ is equal to $11$ units, then value of $\alpha\beta$ is \rule{2cm}{0.2pt}\hfill{\brak{\text{Apr 2023}}}
	\item The coeffecient of $x^{18}$ in the expansion of $\brak{x^4-\frac{1}{x^3}}^{15}$is \rule{2cm}{0.2pt}\hfill{\brak{\text{Apr 2023}}}
	\item Let $A=\cbrak{1,2,3,4,\dots,10}, B=\cbrak{0,1,2,3,4}$. The number of elements in the relation $R=\cbrak{\brak{a,b}\in A \times A: 2\brak{a-b}^2+3\brak{a-b} \in B}$ is \rule{2cm}{0.2pt}\hfill{\brak{\text{Apr 2023}}}
