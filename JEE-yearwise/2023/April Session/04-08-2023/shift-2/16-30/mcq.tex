\iffalse
\title{April 2023, Shift - 2}
\author{EE24BTECH11062}	
\section{mcq-single}
\fi
%\begin{enumerate}
   
\item The integral $\int \brak{\brak{\frac{x}{2}}^x+\brak{\frac{2}{x}}^x}log_2 x\ dx$ is equal to \hfill{[April 2023]}
\begin{multicols}{4}
    a) $\brak{\frac{x}{2}}^x log_2\brak{\frac{2}{x}}+C$\\
    b) $\brak{\frac{x}{2}}^x -\brak{\frac{2}{x}}^x+C$\\
    c) $\brak{\frac{x}{2}}^x log_2\brak{\frac{x}{2}}+C$\\
    d)  $\brak{\frac{x}{2}}^x +\brak{\frac{2}{x}}^x+C$
\end{multicols}
 \item The value of $36\brak{4\cos^2 9\degree -1}\brak{4\cos^2 27\degree -1}\brak{4\cos^2 81\degree}\brak{4\cos^2 243\degree -1}$ is \hfill{[April 2023]}
 \begin{multicols}{4}
     a) 27\\
     b) 54\\
     c) 18\\
     d) 36
 \end{multicols}
 
 \item Let $\vec{A}\brak{0,1}, \vec{B}\brak{1,1}$ and $\vec{C}\brak{1,0}$ be the midpoints of the sides of a triangle with incentre at the point $\vec{D}$. If the focus of the parabola $y^2=4ax$ passing through $\vec{D}$ is $\brak{\alpha +\beta \sqrt{3},0}$, where $\alpha$ and $\beta$ are rational numbers, then $\frac{\alpha}{\beta^2}$ is equal to \hfill{[April 2023]}
 \begin{multicols}{4}
    a) 6\\
    b) 8\\
    c) $\frac{9}{2}$\\
    d) 12
 \end{multicols}
 
\item The negation of $\brak{p \land \brak{\sim q}} \lor \brak{\sim p}
$ is equivalent to \hfill{[April 2023]}
\begin{multicols}{4}
    a) $p \land \brak{\sim q}$\\
    b) $p \land \brak{q \land \brak{\sim p}}$\\
    c)  $p \lor \brak{q \lor \brak{\sim p}}$\\
    d) $p\land q$
\end{multicols}

\item Let the mean and variance of 12 observations be $\frac{9}{2}$ and 4 respectively. Later on, it was observed that two observations were considered as 9 and 10 instead of 7 and 14 respectively. If the correct variance is $\frac{m}{n}$, where $m$ and $n$ are co-prime, then $m+n$ is equal to \hfill{[April 2023]}
\begin{multicols}{4}
    a) 316\\
    b) 317\\
    c) 315\\
    d) 314
\end{multicols}

