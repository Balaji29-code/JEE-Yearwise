\iffalse
\title{Assignment 3}
\author{AI24BTECH11018}
\section{mcq-single}
\fi

%\begin{enumerate}
\item Let the image of the point P\brak{2,-1,3} in the plane $x+2y-z=0$ be $Q$. Then the distance of the plane 
$3x + 2y + z + 29 = 0$ from the point $Q$ is
\hfill{\brak{\text{Jan 2023}}}
\begin{enumerate}
    \item $\frac{22\sqrt{2}}{7}$
    \item $\frac{24\sqrt{2}}{7}$
    \item $2\sqrt{14}$
    \item $3\sqrt{14}$
\end{enumerate}
\item Let $f\brak{x}=\begin{pmatrix}
1+\sin^2{x} & \cos^2{x} & \sin 2x \\
\sin^2{x} & 1+\cos^2{x} & \sin 2x \\
\sin^2{x} & \cos^2{x} & 1+\sin 2x
\end{pmatrix}$, $x\in [\frac{\pi}{6},\frac{\pi}{3}]$.If $\alpha$ a $\beta$ respectively are the maximum and the minimum values of f,then
\hfill{\brak{\text{Jan 2023}}}
\begin{enumerate}
    \item $\beta^2-2\sqrt{\alpha}=\frac{19}{4}$
    \item $\beta^2+2\sqrt{\alpha}=\frac{19}{4}$
    \item $\alpha^2 + \beta^2 = 4\sqrt{3}$
    \item $\alpha^2 + \beta^2 = \frac{9}{2}$
\end{enumerate}
\item Let $f\brak{x}=2x+\tan^{-1}x$ and $g\brak{x}=\log_{e}\brak{\sqrt{1+x^2}+x}$, $x\in [0,3]$ then 
\hfill{\brak{\text{Jan 2023}}}
\begin{enumerate}
    \item There exists $x\in [0,3]$ such that $f^{'}\brak{x} \textless g^{'}\brak{x}$
    \item max $f\brak{x} \textgreater max g\brak{x}$
    \item There exists $0 \textless x_1 \textless x_2\textless 3$ such that $f\brak{x} \textless g\brak{x}$, $\forall x\in \brak{x_1,x_2}$
    \item min $f^{\prime}\brak{x}$=$1$+max $ g^{\prime} \brak{x}$
\end{enumerate}

\item The mean and variance of $5$ observations are $5$ and 
$8$ respectively. If $3$ observations are $1, 3, 5$, then 
the sum of cubes of the remaining two 
observations is
\hfill{\brak{\text{Jan 2023}}}
\begin{enumerate}
    \item $1072$
    \item $1792$
    \item $1216$
    \item $1456$
\end{enumerate}

\item The area enclosed by the closed curve C given by 
the differential equation 
$\frac{dy}{dx}+\frac{x+a}{y-2}=0$, $y\brak{1}=0$ is $4\pi$
 Let $P$ and $Q$ be the points of intersection of the 
curve $C$ and the y-axis. If normals at P and Q on 
the curve $C$ intersect x-axis at points $R$ and $S$ 
respectively, then the length of the line segment 
$RS$ is
\hfill{\brak{\text{Jan 2023}}}
\begin{enumerate}
    \item $2\sqrt{3}$
    \item $\frac{2\sqrt{3}}{3}$
    \item $2$
    \item $\frac{4\sqrt{3}}{3}$
\end{enumerate}

%end{document}
