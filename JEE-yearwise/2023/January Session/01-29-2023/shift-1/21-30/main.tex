\iffalse
\title{2023}   
\author{AI24Btech11024}
\section{integer}              
\fi
\item Let the co-ordinats of one vertex of $\Delta ABC$ be $A\brak{0, 2, \alpha}$ and the other two vertices lie on the line $\frac{x+\alpha}{5}=\frac{y-1}{2}=\frac{z+4}{3}$. For $\alpha\in Z$, if the area of $\Delta ABC$ is 21 sq.units and the line segment BC has length $2\sqrt{21}$ units, then $\alpha^{2}$ is equal to \underline{\hspace{2.5cm}}. \hfill{\brak{\text{Jan 2023}}}

\item Let the equation of the plane p containing the line $x+10=\frac{8-y}{2}=z$ be $ax+by+3z=2\brak{a+b}$ and the distance of the plane P from the point $\brak{1, 27, 7}$ be $c$. Then $a^{2}+b^{2}+c^{2}$ ii equal to \underline{\hspace{2.5cm}}.\hfill{\brak{\text{Jan 2023}}}

\item Suppose f is a function satisfying $f\brak{x+y}=f\brak{x}+f\brak{y}$ for all $x,y \in N$ and $f\brak{1}=\frac{1}{5}$. If $\sum_{n=1}^{m}\frac{f\brak{n}}{n\brak{n+1}\brak{n+2}}=\frac{1}{12}$, then m is equal to \underline{\hspace{2.5cm}}.    \hfill{\brak{\text{Jan 2023}}}

\item Let $a_{1},a_{2},a_{3},\ldots$ be a GP of increasing positive numbers. If the product of fourth and sixth terms is 9 and the sum of fifth and seventh terms is 24, then $a_{1}a_{9}+a_{2}a_{4}a_{9}+a_{5}+a_{7}$ is equal to \underline{\hspace{2.5cm}}.  \hfill{\brak{\text{Jan 2023}}}

\item Let $\overrightarrow{a},\overrightarrow{b}$ and $\overrightarrow{c}$ be three non-coplanar vectors. Let the position vectors of four points $A, B, C and D$ be $\overrightarrow{a}-\overrightarrow{b}+\overrightarrow{c},\lambda\overrightarrow{a}-3\overrightarrow{b}+4\overrightarrow{c},-\overrightarrow{a}+2\overrightarrow{b}-3\overrightarrow{c}$ and $2\overrightarrow{a}-4\overrightarrow{b}+6\overrightarrow{c}$ respectively. If $\overrightarrow{AB},\overrightarrow{AC}$ and $\overrightarrow{AD}$ are coplanar, then $\lambda$ is $\colon$ \hfill{\brak{\text{Jan 2023}}}

\item If all the six digit numbers $X_{1}X_{2}X_{3}X_{4}X_{5}X_{6}$ with $0<X_{1}<X_{2}<X_{3}<X_{4}<X_{5}<X_{6}$ are arranged in the increasing order, then the sum of the digits in the $72^{th}$ number is \underline{\hspace{2.5cm}}. \hfill{\brak{\text{Jan 2023}}}

\item Let $f\colon R \to R$ be a differentiable function that satisfies the relation $f\brak{x+y}=f\brak{x}+f\brak{y}-1$, $\forall{x},y\in R$. If $f^{\prime}\brak{0}=2$, then $\lvert f\brak{-2}\rvert$ is equal to \underline{\hspace{2.5cm}}. \hfill{\brak{\text{Jan 2023}}}

\item If the co-efficient of $x^{9}$ in $\brak{\alpha x^{3}+\frac{1}{\beta x}}^{11}$ and the co-efficient of $x^{-9}$ in $\brak{\alpha x-\frac{1}{\beta x^{3}}}^{11}$ are equal, then $\brak{\alpha\beta}^{2}$ is equal to \underline{\hspace{2.5cm}}. \hfill{\brak{\text{Jan 2023}}}

\item Let the coefficients of three consecutive terms in the binomial expansion of $\brak{1+2x}^{n}$ be in the ratio $2\colon5\colon8$. Then the codfficient of the term, which  is in the middle of these terms, is \underline{\hspace{2.5cm}}. \hfill{\brak{\text{Jan 2023}}}

\item Five digit numbers are formed using the digits 1, 2, 3, 5,7 with repetitions and are written in descending order with serial numbers. For example, the number 77777 has serial number 1. Then the serial number of 35337 is \underline{\hspace{2.5cm}}. \hfill{\brak{\text{Jan 2023}}}

