\iffalse
\title{January 2024, shift 2}
\author{EE24BTECH11062}	
\section{mcq-single}
\fi
%\begin{enumerate}
   
\item Let $\vec{A}$ be the point of intersection of the lines $3x+2y=14, 5x-y=6$ and $\vec{B}$ be the point of intersection of the lines $4x+3y=8, 6x+y=5.$ The distance of the point $\vec{P}\brak{5,-2}$ from the line $AB$ is \hfill{[Jan 2024]}
\begin{multicols}{4}
    a) 8\\
    b) $\frac{5}{2}$\\
    c) 2\\
    d) $\frac{13}{2}$
\end{multicols}
 \item The function $f\brak{x}=\frac{x}{x^2-6x-16}$, $x\in R-\cbrak{-2,8}$ \hfill{[Jan 2024]}
 \begin{enumerate}
     \item  decreases in $\brak{-\infty,-2}$ and increases in $\brak{8,\infty}$\\
     \item decreases in $\brak{-2,8}$ and increases in $\brak{-\infty,-2}\cup \brak{8,\infty}$\\
     \item decreases in $\brak{-\infty,-2}\cup\brak{-2,\infty}\cup\brak{8,\infty}$\\
     \item  increases in $\brak{-\infty,-2}\cup\brak{-2,\infty}\cup\brak{8,\infty}$
 \end{enumerate}
 
 \item If $\sin\brak{\frac{y}{x}}=\ln\abs{x}+\frac{\alpha}{2}$ is the solution of the differential equation $x\cos\brak{\frac{y}{x}}\frac{dy}{dx}=y\cos\brak{\frac{y}{x}}+x$ and $y\brak{1}=\frac{\pi}{3}$, then $\alpha^2$ is equal to \hfill{[Jan 2024]}
 \begin{multicols}{4}
    a) 9\\
    b) 4\\
    c) 12\\
    d) 3
 \end{multicols}
 
\item   If the mean and variance of five observations are $\frac{24}{5}$ and $\frac{194}{25}$ respectively and the mean of the first four observations is $\frac{7}{2}$, then the variance of the first four observations is equal to \hfill{[Jan 2024]}
\begin{multicols}{4}
    a) $\frac{77}{12}$\\
    b) $\frac{105}{4}$\\
    c) $\frac{5}{4}$\\
    d) $\frac{4}{5}$
\end{multicols}

\item Let $r$ and $\theta$ respectively be the modulus and amplitude of the complex number $z=2-i\brak{2\tan \frac{5\pi}{8}}$, then $\brak{r,\theta}$ is equal to \hfill{[Jan 2024]}
\begin{multicols}{4}
    a) $\brak{2\sec \frac{3\pi}{8}, \frac{3\pi}{8}}$\\
    b) $\brak{2\sec \frac{5\pi}{8}, \frac{3\pi}{8}}$\\
    c) $\brak{2\sec \frac{11\pi}{8}, \frac{11\pi}{8}}$\\
    d) $\brak{2\sec \frac{3\pi}{8}, \frac{5\pi}{8}}$
\end{multicols}

