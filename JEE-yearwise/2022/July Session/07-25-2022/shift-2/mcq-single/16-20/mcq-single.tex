
\iffalse
  \title{2022}
  \author{EE24BTECH11043}
  \section{mcq-single}
\fi
\item The shortest distance between the lines $\frac{x+7}{-6} = \frac{y-6}{7} = z$ and $\frac{7-x}{2} = y-2 = z-6$ is \hfill{\sbrak{JUL 2022}}
\begin{enumerate}
\item $2\sqrt{29}$
\item $1$
\item $\sqrt{\frac{37}{29}}$
\item $\sqrt{\frac{29}{2}}$
\end{enumerate}
\item Let $\vec{a} = \hat{i}-\hat{j}+2\hat{k}$ and let $\vec{b}$ be a vector such that $\vec{a} \times \vec{b} = 2\hat{i}-\hat{k}$ and $\vec{a} \cdot \vec{b} = 3$ Then the projection of $\vec{b}$ on the vector $\vec{a}-\vec{b}$ is : \hfill{\sbrak{JUL 2022}}
\begin{enumerate}
\item $\frac{2}{\sqrt{21}}$
\item $2\sqrt{\frac{3}{7}}$
\item $\frac{2}{3}\sqrt{\frac{7}{3}}$
\item $\frac{2}{3}$
\end{enumerate}
\item If the mean deviation about median for the number $3,5,7,2k,12,16,21,24$ arranged in the ascending order, is 6 then the median is \hfill{\sbrak{JUL 2022}}
\begin{enumerate}
\item $11.5$
\item $10.5$
\item $12$
\item $11$
\end{enumerate}
\item $2\sin\brak{\frac{\pi}{22}}\sin\brak{\frac{3\pi}{22}}\sin\brak{\frac{5\pi}{22}}\sin\brak{\frac{7\pi}{22}}\sin\brak{\frac{9\pi}{22}}$ is eqaul to : \hfill{\sbrak{JUL 2022}}
\begin{enumerate}
\item $\frac{3}{16}$
\item $\frac{1}{16}$
\item $\frac{1}{32}$
\item $\frac{9}{32}$
\end{enumerate}
\item Consider the following statements : \hfill{\sbrak{JUL 2022}}\\
$P$ : Ramu is intelligent. \\
$Q$ : Ramu is rich. \\
$R$ : Ramu is not honest.\\
The negation of the statement "Ramu is intelligent and honest if and only if Ramu is not rich" can be expressed as : 
\begin{enumerate}
\item $\brak{\brak{P \cap \brak{\sim R}}\cap Q} \brak{\brak{\sim Q} \cap \brak{\brak{\sim P} \cup R}}$
\item $\brak{\brak{P \cap R} \cap Q} \cup\brak{\brak{\sim Q} \cap \brak{\brak{\sim P} \cup \brak{\sim R}}}$
\item $\brak{\brak{P \cap R} \cap Q} \cap \brak{\brak{\sim Q} \cap \brak{\brak{\sim P} \cup \brak{\sim R}}}$
\item $\brak{\brak{P \cap \brak{\sim R}} \cap Q} \cup \brak{\brak{\sim Q} \cap \brak{\brak{\sim P} \cap R}}$
\end{enumerate}

