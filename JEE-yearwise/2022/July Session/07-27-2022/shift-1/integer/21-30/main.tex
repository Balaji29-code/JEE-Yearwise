\iffalse
	\title{2022}
	\author{AI24BTECH11003}
	\section{integer}
\fi

%21
    \item For $k\in R$, let the solutions of the equation $\cos\brak{\arcsin\brak{x\cot\brak{\arctan\brak{\cos\brak{\arcsin x}}}}}=k,0<\abs{x}<\frac{1}{\sqrt{2}}$ be $\alpha$ and $\beta$, where the inverse trigonometric functions take only principal values. If the solutions of the equation $x^2-bx-5=0$ are $\frac{1}{\alpha^2}+\frac{1}{\beta^2}$ and $\frac{\alpha}{\beta}$, then $\frac{b}{k^2}$ is equal to
    
    \hfill[Jul 2022]

%22
    \item The mean and variance of 10 observations were calculated as 15 and 15 respectively by a student who took by mistake 25 instead of 15 for one observation. Then the correct standard deviation is
    
    \hfill[Jul 2022]
		
%23
    \item Let the line $\frac{x-3}{7}=\frac{y-2}{-1}=\frac{z-3}{-4}$ intersect the plane containing the lines $\frac{x-4}{1}=\frac{y+1}{-2}=\frac{z}{1}$ and $4ax-y+5z-7a=0=2x-5y-z-3,a\in R$ at the point P$\brak{\alpha,\beta,\gamma}$. Then, the value of $\alpha+\beta+\gamma$ equals
    
    \hfill[Jul 2022]

%24
    \item An ellipse $E:\frac{x^2}{a^2}+\frac{y^2}{b^2}=1$ passes through the vertices of the hyperbola $H:\frac{x^2}{49}-\frac{y^2}{64}=-1$. Let the major and minor axes of the ellipse E coincide with the transverse and conjugate axes of the hyperbola $H$. Let the product of the eccentricities of $E$ and $H$ be $\frac{1}{2}$. If $l$ is the length of the latus rectum of the ellipse $E$, then the value of 113l is equal to:
    
    \hfill[Jul 2022]

%25
    \item Let $y=y\brak{x}$ be the solution curve of the differential equation $\sin\brak{2x^2}\log_e\brak{\tan x^2}dy+\brak{4xy-4\sqrt{2}x\sin\brak{x^2-\frac{\pi}{4}}}dx=0, 0<x<\sqrt{\frac{\pi}{2}}$, which passes through the point $\brak{\sqrt{\frac{\pi}{6}},1}$. Then $\abs{y\brak{\sqrt{\frac{\pi}{3}}}}$ is equal to
    
    \hfill[Jul 2022]
        
%26
    \item Let $M$ and $N$ be the number of points on the curve $y^5-9xy+2x=0$, where the tangents on the curve are parallel to x-axis and y-axis, respectively. Then the value of $M+N$ equals
    
    \hfill[Jul 2022]

%27
    \item Let $f\brak{x}=2x^2-x-1$ and $S=\left\{ n\in Z:\abs{f\brak{n}}\leq800 \right\}$. Then, the value of $\underset{n\in S}{\sum}f\brak{n}$ is equal to
    
    \hfill[Jul 2022]
        
%28
    \item Let S be the set containing all 3$\times$3 matrices with entries from $\left\{ -1,0,1 \right\}$. The total number of matrices $A\in S$ such that the sum of all the diagonal elements of $A^\top A$ is 6 is
    
    \hfill[Jul 2022]

%29
    \item If the length of the latus rectum of the ellipse $x^2+4y^2x+8y-\lambda=0$ is 4, and $l$ is the length of its major axis, then $\lambda+l$ is equal to
    
    \hfill[Jul 2022]

%30
    \item Let $S=\left\{ z\in C:z^2+\Bar{z}=0 \right\}$. Then $\underset{z\in S}{\sum}\brak{Re\brak{z}+Im\brak{z}}$ is equal to
    
    \hfill[Jul 2022]

