 \iffalse
  \title{2022}
  \author{EE24BTECH11010}
  \section{mcq-single}
\fi
 \item Let $R$ be a relation from the set $\cbrak{1,2,3,...,60}$
 to itself such that $R = \cbrak{\brak{a,b} : b = pq, \text{ where } p, q \ge 3 \text{ are prime numbers }}$. Then, the  number of elements in $R$ is : \hfill [July 2022]
 \begin{enumerate}
     \begin{multicols}{2}
         \item 600
         \item 660
         \item 540
         \item 720
     \end{multicols}
 \end{enumerate}
 \item $z = 2 + 3i$, then $z^5 + \brak{\bar{z}}^5$ is equal to: \hfill [July 2022]
 \begin{enumerate}
     \begin{multicols}{2}
         \item 244
         \item 224
         \item 245
         \item 265
     \end{multicols}
 \end{enumerate}
 \item Let $A$ and $B$ be two $3 \times 3$
 non-zero real matrices such that $AB$ is a zero matrix. Then

 \hfill[July 2022]
 \begin{enumerate}
     
         \item the system of linear equation $AX = 0$ has a unique solution
         \item the system of linear equation $AX = 0$ has infinitely many solutions
         \item $B$ is an invertible matrix
	 \item $\operatorname{adj} \brak{A}$ is an invertible matrix 
 \end{enumerate}
 \item If $\frac{1}{\brak{20-a}\brak{40-a}} + \frac{1}{\brak{40-a}\brak{60-a}} + . . . . + \frac{1}{\brak{180-a}\brak{200-a}} = \frac{1}{256}$, then the maximum value of $a$ is:

 \hfill[July 2022]
 \begin{enumerate}
     \begin{multicols}{2}
         \item 198
         \item 202
         \item 212
         \item 218
     \end{multicols}
 \end{enumerate}
 \item If $\lim_{x\to 0} \frac{\alpha e^x + \beta e^{-x} + \gamma \sin{x}}{x \sin ^2 x} = \frac{2}{3}$, where $\alpha, \beta, \gamma \in \mathbb{R}$, then which of the following is NOT correct? \hfill [July 2022]
 \begin{enumerate}
     \begin{multicols}{2}
         \item $\alpha ^2 + \beta ^2 + \gamma ^2 = 6$
         \item $\alpha\beta + \beta\gamma + \gamma\alpha + 1 = 0$
         \item $\alpha\beta ^2 + \beta \gamma ^2 + \gamma\alpha ^2 + 3 = 0$
         \item $\alpha^2 - \beta ^2 + \gamma^2 =4$
     \end{multicols}
 \end{enumerate}
 \item The integral $\int_0^{\frac{\pi}{2}} \frac{1}{3 + 2 \sin{x} + \cos{x}}dx$ is equal to: \hfill [July 2022]
 \begin{enumerate}
     \begin{multicols}{4}
         \item $\tan^{-1}\brak{2}$
         \item $\tan^{-1}\brak{2} - \frac{\pi}{4}$
         \item $\frac{1}{2}\tan^{-1}\brak{2} - \frac{\pi}{8}$
         \item $\frac{1}{2}$
     \end{multicols}
 \end{enumerate}
 \item Let the solution curve $y = y\brak{x}$
 of the differential equation $\brak{1 + e^{2x}}\brak{\frac{dy}{dx} + y} = 1$
 pass through the point $\brak{0,\frac{\pi}{2}}$ 
. Then,  $\lim_{x \to \infty} e^x y\brak{x}$ is equal to: \hfill[July 2022]
\begin{enumerate}
    \begin{multicols}{4}
        \item $\frac{\pi}{4}$
        \item $\frac{3\pi}{4}$
        \item $\frac{\pi}{2}$
        \item $\frac{3\pi}{2}$
    \end{multicols}
\end{enumerate}
\item Let a line $L$ pass through the point of intersection of the lines $bx + 10y - 8 = 0$ and  $2x - 3y = 0, b \in \mathbb{R} - \cbrak{\frac{4}{3}}$
. If the line $L$
 also passes through the point $\brak{1,1}$
 and touches the circle $17\brak{x^2 + y^2} = 16$
, then the eccentricity of the ellipse $\frac{x^2}{5} + \frac{y^2}{b^2} = 1$ 
 is : 

 \hfill [July 2022]
 \begin{enumerate}
     \begin{multicols}{4}
     \item $\frac{2}{\sqrt{5}}$
     \item $\sqrt{\frac{3}{5}}$
     \item $\frac{1}{\sqrt{5}}$
     \item $\sqrt{\frac{2}{5}}$
     \end{multicols}
 \end{enumerate}
 \item If the foot of perpendicular from the point $A\brak{-1,4,3}$ on the plane $P : 2x + my + nz = 4$ is $\brak{-2, \frac{7}{2}, \frac{3}{2}}$, then the distance of the point $A$ from the plane $P$, measured parallel to a line with direction ratios $3, -1, -4$, is equal to: \hfill[July 2022]
 \begin{enumerate}
     \begin{multicols}{2}
         \item 1
         \item $\sqrt{26}$
         \item $2\sqrt{2}$
         \item $\sqrt{14}$
     \end{multicols}
 \end{enumerate}
 \item Let $\vec{a} = 3\hat{i} + \hat{j}$ and $\vec{b} = \hat{i} + 2\hat{j} + \hat{k}$. Let $\vec{c}$ be a vector satisfying $\vec{a} \times \brak{\vec{b} \times \vec{c}} = \vec{b} + \lambda \vec{c}$. If $\vec{b}$ and $\vec{c}$ are non-parallel, then the value of $\lambda$ is: \hfill [July 2022]
 \begin{enumerate}
     \begin{multicols}{2}
         \item $-5$
         \item 5
         \item 1
         \item $-1$
     \end{multicols}
 \end{enumerate}
 \item The angle of elevation of the top of a tower from a point $A$ due north of it is $\alpha$ and from a point $B$ at a distance of 9 units due west of $A$ is $\cos^{-1}\brak{\frac{3}{\sqrt{13}}}$. If the distance of the point $B$ from the tower is 15 units, then $\cot{\alpha}$ is equal to : \hfill [July 2022]
 \begin{enumerate}
     \begin{multicols}{4}
         \item $\frac{6}{5}$
         \item $\frac{9}{5}$
         \item $\frac{4}{3}$
         \item $\frac{7}{3}$
     \end{multicols}
 \end{enumerate}
 \item The statement $\brak{p \wedge q} \implies \brak{p \wedge r}$
 is equivalent to: \hfill [July 2022]
 \begin{enumerate}
     \begin{multicols}{2}
         \item $q \implies \brak{p \wedge r}$
         \item $p \implies \brak{p \wedge r}$
         \item $\brak{p \wedge r} \implies \brak{p \wedge q}$
         \item $\brak{p \wedge q} \implies r$
     \end{multicols}
 \end{enumerate}
 \item Let the circumcentre of a triangle with vertices $A(a, 3), B(b, 5) \text{ and } C(a, b), ab > 0 $ be $P(1,1)$. If the line $AP$ intersects the line $ BC$ at the point $Q\brak{k_1,k_2}$
, then $k_1 + k_2$ is equal to : \hfill [July 2022]
\begin{enumerate}
    \begin{multicols}{2}
        \item 2
        \item $\frac{4}{7}$
        \item $\frac{2}{7}$
        \item 4
    \end{multicols}
\end{enumerate}
\item Let $\hat{a}$
 and $\hat{b}$
 be two unit vectors such that the angle between them is 
 $\frac{\pi}{4}$
. If $\theta$
 is the angle between the vectors $\brak{\hat{a} + \hat{b}}$
 and $\brak{\hat{a} + 2\hat{b} + 2\brak{\hat{a}\times \hat{b}}}$
, then the value of $164 \cos^2\theta$
 is equal to : \hfill [July 2022]
 \begin{enumerate}
     \begin{multicols}{2}
         \item $90 + 27 \sqrt{2}$
         \item $45 + 18\sqrt{2}$
         \item $90 + 3\sqrt{2}$
         \item  $54 + 90\sqrt{2}$
     \end{multicols}
 \end{enumerate}
 \item $f\brak{\alpha} = \int_1^{\alpha} \frac{\log_{10}t}{1 + t}dt, \alpha > 0$, then $f\brak{e^3} + f\brak{e^{-3}}$ is equal to: \hfill [July 2022]
 \begin{enumerate}
     \begin{multicols}{4}
     \item 9
     \item $\frac{9}{2}$
    \item $\frac{9}{\log_{e}\brak{10}}$
    \item $\frac{9}{2\log_{e}\brak{10}}$
     \end{multicols}
 \end{enumerate}