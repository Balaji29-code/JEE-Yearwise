\iffalse
	\title{2022}
	\author{AI24BTECH11003}
	\section{mcq-single}
\fi

%1
    \item Let $S=\left\{ x\in \left[-6,3\right]-\left\{-2,2\right\}:\frac{\abs{x+3}-1}{\abs{x}-2} \geq 0\right\}$ and $T=\left\{ x\in Z:x^2-7\abs{x}+9\leq0\right\}$. Then the number of elements in $S\cap T$ is
    
    \hfill[Jul 2022]

        \begin{multicols}{4}
            \begin{enumerate}
                \item $7$
                \item $5$
                \item $4$
                \item $3$
            \end{enumerate}
        \end{multicols}

%2
    \item Let $\alpha,\beta$ be roots of the equation $x^2-\sqrt{2}x+\sqrt{6}=0$ and $\frac{1}{\alpha^2}+1,\frac{1}{\beta^2}+1$ be the roots of the equation $x^2+ax+b=0$. Then the roots of the equation $x^2-\brak{a+b-2}x+\brak{a+b+2}=0$ are:
    
    \hfill[Jul 2022]

		\begin{multicols}{2}
			\begin{enumerate}
				\item non-real complex numbers
				\item real and both negative
				\item real and both positive
				\item real and exactly one of them is positive
			\end{enumerate}
		\end{multicols}

%3
    \item Let A and B be any two 3$\times$3 symmetric and skew-symmetric matrices respectively. Then which of the following is \underline{NOT} true?
    
    \hfill[Jul 2022]

        \begin{multicols}{2}
            \begin{enumerate}
                \item A$^4 -$ B$^4$ is a symmetric matrix
                \item AB$-$BA is a symmetric matrix
                \item B$^5-$A$^5$ is a skew-symmetric matrix
                \item AB$+$BA is a skew-symmetric matrix
            \end{enumerate}
        \end{multicols}

%4
    \item Let $f\brak{x}=ax^2+bx+c$ be such that $f\brak{1}=3,f\brak{-2}=\lambda$ and $f\brak{3}=4$. If $f\brak{0}+f\brak{1}+f\brak{-2}+f\brak{3}=14$, then $\lambda$ is equal to 
    
    \hfill[Jul 2022]

		\begin{multicols}{4}
			\begin{enumerate}
				\item $-4$
				\item $\frac{13}{2}$
				\item $\frac{23}{2}$
				\item $4$
			\end{enumerate}
		\end{multicols}

%5
    \item The function $f:R\to R$ defined by $f\brak{x}=\underset{n\to\infty}{\lim}\frac{\cos\brak{2\pi x}-x^{2n}\sin\brak{x-1}}{1+x^{2n+1}-x^{2n}}$ is continuous for all x in
    
    \hfill[Jul 2022]

		\begin{multicols}{4}
			\begin{enumerate}
				\item $R-\left\{-1\right\}$
				\item $R-\left\{-1,1\right\}$
				\item $R-\left\{1\right\}$
				\item $R-\left\{0\right\}$
			\end{enumerate}
		\end{multicols}
  
%6
    \item The function $f\brak{x}=xe^{x\brak{1-x}},x\in R$, is
    
    \hfill[Jul 2022]

        \begin{multicols}{2}
            \begin{enumerate}
                \item increasing in $\brak{-\frac{1}{2},1}$
                \item decreasing in $\brak{\frac{1}{2},2}$
                \item increasing in $\brak{-1,-\frac{1}{2}}$
                \item decreasing in $\brak{-\frac{1}{2},\frac{1}{2}}$
            \end{enumerate}
        \end{multicols}

%7
    \item The sum of the absolute maximum and absolute minimum values of the function $f\brak{x}=\arctan\brak{\sin x-\cos x}$ in the interval $\left[0,\pi\right]$ is
    
    \hfill[Jul 2022]

        \begin{multicols}{4}
            \begin{enumerate}
                \item 0
                \item $\arctan\brak{\frac{1}{\sqrt{2}}}-\frac{\pi}{4}$
                \item $\arccos\brak{\frac{1}{\sqrt{3}}}-\frac{\pi}{4}$
                \item $\frac{\pi}{4}$
            \end{enumerate}
        \end{multicols}
		
%8
    \item Let $x\brak{t}=2\sqrt{2}\cos{t}\sqrt{\sin{2t}}$ and $y\brak{t}=2\sqrt{2}\sin{t}\sqrt{\sin{2t}},t\in\brak{0,\frac{\pi}{2}}$. Then $\frac{1+\brak{\frac{dy}{dx}}^2}{\frac{d^2y}{dx^2}}$ at $t=\frac{\pi}{4}$ is equal to
    
    \hfill[Jul 2022]

        \begin{multicols}{4}
            \begin{enumerate}
                \item $\frac{-2\sqrt{2}}{3}$
                \item $\frac{2}{3}$
                \item $\frac{1}{3}$
                \item $\frac{-2}{3}$
            \end{enumerate}
        \end{multicols}

%9
    \item Let $I_n\brak{x}=\int_0^x\frac{1}{\brak{t^2+5}^n}dt,n=1,2,3\cdots$ Then 
    
    \hfill[Jul 2022]

        \begin{multicols}{4}
            \begin{enumerate}
                \item $50I_6-9I_5=xI'_5$
                \item $50I_6-11I_5=xI'_5$
                \item $50I_6-9I_5=I'_5$
                \item $50I_6-11I_5=I'_5$
            \end{enumerate}
        \end{multicols}

%10
    \item The area enclosed by the curves $y=\log_e\brak{x+e^2},x=\log_e\brak{\frac{2}{y}}$ and $x=\log_e2$, above the line $y=1$ is
    
    \hfill[Jul 2022]

        \begin{multicols}{4}
            \begin{enumerate}
                \item $2+e-\log_e2$
                \item $1+e-\log_e2$
                \item $e-\log_e2$
                \item $1+\log_e2$
            \end{enumerate}
        \end{multicols}
        
%11
    \item Let $y=y\brak{x}$ be the solution curve of the differential equation $\frac{dy}{dx}+\frac{1}{x^2-1}y=\brak{\frac{x-1}{x+1}}^{\frac{1}{2}},x>1$ passing through the point $\brak{2,\sqrt{\frac{1}{3}}}$. Then $\sqrt{7}y\brak{8}$ is equal to
    
    \hfill[Jul 2022]

        \begin{multicols}{4}
            \begin{enumerate}
                \item $11+6\log_e3$
                \item $19$
                \item $12-2\log_e3$
                \item $19-6\log_e3$
            \end{enumerate}
        \end{multicols}

%12
    \item The differential equation of the family of circles passing through the points $\brak{0,2}$ and $\brak{0,-2}$ is
    
    \hfill[Jul 2022]

        \begin{multicols}{2}
            \begin{enumerate}
                \item $2xy\frac{dy}{dx}+\brak{x^2-y^2+4}=0$
                \item $2xy\frac{dy}{dx}+\brak{x^2+y^-+4}=0$
                \item $2xy\frac{dy}{dx}+\brak{y^2-x^2+4}=0$
                \item $2xy\frac{dy}{dx}-\brak{x^2-y^2+4}=0$
            \end{enumerate}
        \end{multicols}
        
%13
    \item Let the tangents at two points $A$ and $B$ on the circle $x^2+y^2-4x+3=0$ meet at the origin $O\brak{0,0}$. Then the area of the triangle $OAB$ is
    
    \hfill[Jul 2022]

        \begin{multicols}{4}
            \begin{enumerate}
                \item $\frac{3\sqrt{3}}{2}$
                \item $\frac{3\sqrt{3}}{4}$
                \item $\frac{3}{2\sqrt{3}}$
                \item $\frac{3}{4\sqrt{3}}$
            \end{enumerate}
        \end{multicols}

%14
    \item Let the hyperbola $H:\frac{x^2}{a^2}-\frac{y^2}{b^2}=1$ pass through the point $\brak{2\sqrt{2},-2\sqrt{2}}$. A parabola is drawn whose focus is same as the focus of H with positive abscissa and the directrix of the parabola passes through the other focus of $H$. If the length of the latus rectum of the parabola is $e$ times the length of the latus rectum of $H$, where $e$ is the eccentricity of $H$, then which of the following points lies on the parabola?
    
    \hfill[Jul 2022]

        \begin{multicols}{4}
            \begin{enumerate}
                \item $\brak{2\sqrt{3},3\sqrt{2}}$
                \item $\brak{3\sqrt{3},-6\sqrt{2}}$
                \item $\brak{\sqrt{3},-\sqrt{6}}$
                \item $\brak{3\sqrt{6},6\sqrt{2}}$
            \end{enumerate}
        \end{multicols}

%15
    \item Let the lines $\frac{x-1}{\lambda}=\frac{y-2}{1}=\frac{z-3}{2}$ and $\frac{x+26}{-2}=\frac{y+18}{3}=\frac{z+28}{\lambda}$ be coplanar and P be the plane containing these two lines. Then which of the following points does NOT lie on P?
    
    \hfill[Jul 2022]

        \begin{multicols}{4}
            \begin{enumerate}
                \item $\brak{0, -2, -2}$
                \item $\brak{-5, 0, -1}$
                \item $\brak{3, -1, 0}$
                \item $\brak{0, 4, 5}$
            \end{enumerate}
        \end{multicols}

