\iffalse
    \title{2022}
    \author{EE24BTECH11005}
    \section{mcq-single}
\fi 
	\item If $\sum_{k=1}^{31}\binom{31}{k}\binom{31}{k-1}-\sum_{k=1}^{30} \binom{30}{k}\binom{30}{k-1}=\frac{\alpha\brak{60!}}{\brak{30!}\brak{31!}}$, where $\alpha \in R$, then the value of $16\alpha$ is equal to,
		\hfill{\brak{2022-April}}\\ \begin{enumerate}
				\begin{multicols}{2}
					\item $1411$
				\columnbreak
					\item $1320$
				\end{multicols}
				\begin{multicols}{2}
					\item $1615$
				\columnbreak
					\item $1855$
				\end{multicols}
		\end{enumerate}
	\item Let a function $f N\to N$ be defined by,
		\begin{align*}
			f\brak{n}=
			\begin{cases}
				2n, &n=2,4,6,8,\dots\\
				n-1, &n=3,7,11,15,\dots\\
				\frac{n+1}{2}, &n=1,5,9,13,\dots
			\end{cases}
		\end{align*}
		then $f$ is,
		\hfill{\brak{2022-April}}\\ \begin{enumerate}
				\begin{multicols}{2}
					\item one-one but not onto
				\columnbreak
					\item onto but not one-one
				\end{multicols}
				\begin{multicols}{2}
					\item neither one-one nor onto
				\columnbreak
					\item one-one and onto
				\end{multicols}
		\end{enumerate}
	\item If the system of linear equations,
		\begin{align*}
			2x+3y-z=-2\\
			x+y+z=4\\
			x-y+\abs{\lambda}z=4\lambda-4
		\end{align*}
		where $\lambda \in R$, has no solution then,
		\hfill{\brak{2022-April}}\\ \begin{enumerate}
				\begin{multicols}{2}
				\item $\lambda=7$
				\columnbreak
			\item $\lambda=-7$
				\end{multicols}
				\begin{multicols}{2}
				\item $\lambda=8$
				\columnbreak
			\item $\lambda^2=1$
				\end{multicols}
		\end{enumerate}
	\item Let $A$ be a matrix of order $3\times3, \abs{A}=2$. Then
		\begin{align*}
			\abs{\abs{A}adj\brak{5adj\brak{A^3}}}
		\end{align*}
		is equal to,
		\hfill{\brak{2022-April}}\\ \begin{enumerate}
				\begin{multicols}{2}
					\item $512\times10^6$
				\columnbreak
					\item $256\times10^6$
				\end{multicols}
				\begin{multicols}{2}
					\item $1024\times10^6$
				\columnbreak
			\item $256\times10^{11}$
				\end{multicols}
		\end{enumerate}
	\item The total number of 5 digit numbers, formed by using the digits $1,2,3,5,6,7$ without repitition which are a multiple of 6 is,
		\hfill{\brak{2022-April}}\\ \begin{enumerate}
				\begin{multicols}{2}
				\item $36$
				\columnbreak
			\item $48$
				\end{multicols}
				\begin{multicols}{2}
				\item $50$
				\columnbreak
			\item $72$
				\end{multicols}
			\end{enumerate}
		\item Let $A_1,A_2,A_3,\dots$ be an increasing geometric progression of positive real numbers. If $A_1A_3A_5A_7=\frac{1}{1296}$, and $A_2+A_4=\frac{7}{36}$, then the value of $A_6+A_8+A_{10}$ is,
		\hfill{\brak{2022-April}}\\ \begin{enumerate}
				\begin{multicols}{2}
					\item $33$
				\columnbreak
					\item $37$
				\end{multicols}
				\begin{multicols}{2}
					\item $43$
				\columnbreak
					\item $47$
				\end{multicols}
		\end{enumerate}
	\item Let $\sbrak{t}$ denote the greatest integer less than or equal to $t$. Then, the value of the integral $\int _0 ^1\sbrak{-8x^2+6x-1}dx$ is equal to,
		\hfill{\brak{2022-April}}\\ \begin{enumerate}
				\begin{multicols}{2}
					\item $-1$
				\columnbreak
			\item $-\frac{5}{4}$
				\end{multicols}
				\begin{multicols}{2}
				\item $\frac{\sqrt{17}-13}{8}$
				\columnbreak
			\item $\frac{\sqrt{17}-16}{8}$
				\end{multicols}
		\end{enumerate}
	\item Let $f:R\to R$ be defined as,
		\begin{align*}
			f\brak{x}=
	\begin{cases}
		\sbrak{e^x}, &x<0\\
				ae^x+\sbrak{x-1}, & 0 \le x < 1\\
				b+\sbrak{\sin\brak{\pi x}}, & 1 \le x < 2\\
				\sbrak{e^{-x}}-c, & x \ge 2
			\end{cases}
		\end{align*}
		where $a,b,c \in R$ and $\sbrak{t}$ denotes greatest integer less than or equal to $t$. Then, which of the following statements is true? 
		\hfill{\brak{2022-April}}\\ \begin{enumerate}
				\begin{multicols}{2}
				\item There exists $a,b,c \in R$ such that $f$ is\\ continous of $R$
				\columnbreak
			\item If $f$ is discontinous at exactly one point, $a+b+c=1$
				\end{multicols}
				\begin{multicols}{2}
				\item If $f$ is discontinous at exactly one \\point, $a+b+c \neq 1$
				\columnbreak
			\item $f$ is discontinous at atleast two points for any values of $a,b,c$
				\end{multicols}
		\end{enumerate}
	\item The area of the region
		\begin{align*}
			S=\cbrak{\brak{x,y}:y^2\le8x,y\ge\sqrt{2}x,x\ge1}
		\end{align*}
		is,
		\hfill{\brak{2022-April}}\\ \begin{enumerate}
				\begin{multicols}{2}
				\item $\frac{13\sqrt{2}}{6}$
				\columnbreak
			\item $\frac{11\sqrt{2}}{6}$
				\end{multicols}
				\begin{multicols}{2}
				\item $\frac{5\sqrt{2}}{6}$
				\columnbreak
			\item $\frac{19\sqrt{2}}{6}$
				\end{multicols}
		\end{enumerate}
	\item Let the solution curve $y=f\brak{x}$ of the differential equation,
		\begin{align*}
			\sbrak{\frac{x}{\sqrt{x^2-y^2}}+e^{\frac{y}{x}}}x\frac{dy}{dx}=x+\sbrak{\frac{x}{\sqrt{x^2-y^2}}+e^{\frac{y}{x}}}y
		\end{align*}
		pass through the points $\myvec{1\\0},\myvec{2\alpha\\\alpha},\alpha>0$. Then $\alpha$ is equal to,
		\hfill{\brak{2022-April}}\\ \begin{enumerate}
				\begin{multicols}{2}
				\item $\frac{1}{2}\exp\brak{\frac{\pi}{6}+\sqrt{e}-1}$
				\columnbreak
				\item $\frac{1}{2}\exp\brak{\frac{\pi}{3}+\sqrt{e}-1}$
				\end{multicols}
				\begin{multicols}{2}
			\item $\exp\brak{\frac{\pi}{6}+\sqrt{e}-1}$
				\columnbreak
				\item $2\exp\brak{\frac{\pi}{6}+\sqrt{e}-1}$
				\end{multicols}
			\end{enumerate}
		\item Let  $y=y\brak{x}$ be the solution to the differential equation $x\brak{1-x^2}\frac{dy}{dx}+\brak{3x^2y-y-4x^3}=0,x>1$ with $y\brak{2}=-2$. Then $y\brak{3}$ is equal to
		\hfill{\brak{2022-April}}\\ \begin{enumerate}
				\begin{multicols}{2}
					\item $-18$
				\columnbreak
					\item $-12$
				\end{multicols}
				\begin{multicols}{2}
					\item $-6$
				\columnbreak
					\item $-3$
				\end{multicols}
		\end{enumerate}
	\item The number of real solutions of $x^7+5x^3+3x+1=0$ is equal to,
			\hfill{\brak{2022-April}}\\ \begin{enumerate}
				\begin{multicols}{2}
					\item $0$
				\columnbreak
					\item $1$
				\end{multicols}
				\begin{multicols}{2}
					\item $3$
				\columnbreak
					\item $5$
				\end{multicols}
		\end{enumerate}
	\item Let the eccentricity of the hyperbola $H: \frac{x^2}{a^2}-\frac{y^2}{b^2}=1$ be $\sqrt{\frac{5}{2}}$ and the length of the latus rectum be $6\sqrt{2}$. If $y=2x+c$ is a tangent to hyperbola $H$, then the value of $c^2$ is equal to,
		\hfill{\brak{2022-April}}\\ \begin{enumerate}
				\begin{multicols}{2}
				\item $18$
				\columnbreak
			\item $20$
				\end{multicols}
				\begin{multicols}{2}
				\item $24$
				\columnbreak
			\item $32$
				\end{multicols}
		\end{enumerate}
	\item If the tangents drawn at the point $O\myvec{0\\0}, P\myvec{1+\sqrt{5}\\2}$ on the circle $x^2+y^2-2x-4y=0$ intersect at point $Q$, then the area of triangle $OPQ$ is equal to
		\hfill{\brak{2022-April}}\\ \begin{enumerate}
				\begin{multicols}{2}
				\item $\frac{3+\sqrt{5}}{2}$
				\columnbreak
			\item $\frac{4+2\sqrt{5}}{2}$
				\end{multicols}
				\begin{multicols}{2}
				\item $\frac{5+3\sqrt{5}}{2}$
				\columnbreak
			\item $\frac{7+3\sqrt{5}}{2}$
				\end{multicols}
		\end{enumerate}
	\item If two distinct points $Q,R$ lie on the line of intersection of the planes $-x+2y-z=0$ and $3x-5y+2z=0$ and $PQ=PR=\sqrt{18}$ where the point $P$ is $\myvec{1\\-2\\3}$, then the area of the triangle $PQR$ is equal to
		\hfill{\brak{2022-April}}\\ \begin{enumerate}
				\begin{multicols}{2}
				\item $\frac{2}{3}\sqrt{38}$
				\columnbreak
			\item $\frac{4}{3}\sqrt{38}$
				\end{multicols}
				\begin{multicols}{2}
				\item $\frac{8}{3}\sqrt{38}$
				\columnbreak
			\item $\sqrt{\frac{152}{3}}$
				\end{multicols}
			\end{enumerate}

