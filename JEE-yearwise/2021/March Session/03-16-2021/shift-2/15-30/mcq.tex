\iffalse
    \title{2021}
    \author{EE24BTECH11002}
    \section{mcq-single}
\fi

    %Question16
    \item Let $C_1$ be the curve obtained by the solution of the differential equation $2xy\frac{dy}{dx} = y^2 - x^2, x > 0$. Let curve $C_2$ be the solution of $\frac{2xy}{x^2 - y^2} = \frac{dy}{dx}$. If both the curves pass through $\brak{1, 1}$, then the area enclosed by the curves $C_1$ and $C_2$ is equal to
    \hfill{\brak{\text{Mar 2021}}}

	\begin{enumerate}
		\item $\frac{\pi}{2} - 1$
		\item $\frac{\pi}{4} + 1$
		\item $\pi - 1$
		\item $\pi + 1$
	\end{enumerate}

    %Question17
    \item Let $\vec{a} = \vec{i} + 2\vec{j} - 3\vec{k}$ and $\vec{b} = 2\vec{i} - 3\vec{j} + 5\vec{k}$. If $\vec{r} \times \vec{a} = \vec{b} \times \vec{r}$, $\vec{r} \cdot \brak{\alpha\vec{i} + 2\vec{j} + \vec{k}} = 3$ and $\vec{r} \cdot \brak{2\vec{i} + 5\vec{j} - \alpha\vec{k}} = -1, \alpha \in \mathbb{R}$, then the value of $\alpha + \abs{r}^2 = $
    \hfill{\brak{\text{Mar 2021}}}

	\begin{enumerate}
		\item $11$ 
		\item $15$
		\item $9$
		\item $13$
	\end{enumerate}

    %Question18
    \item Let $P\brak{x} = x^2 + bx + c$ be a quadratic polynomial with real coefficients such that $\int_{0}^{1} P\brak{x} \, dx$ and $P\brak{x}$ leaves remainder 5 when divided by $\brak{x - 2}$. Then the value of $9\brak{b+c}$ is equal to
    \hfill{\brak{\text{Mar 2021}}}

	\begin{enumerate}
		\item $7$ 
		\item $11$
		\item $15$
		\item $9$
	\end{enumerate}


    %Question19
    \item If the points of intersections of the ellipse $\frac{x^2}{16} + \frac{y^2}{b^2} = 1$ and the circle $x^2 + y^2 = 4b, b > 4$ lie on the curve $y^2 = 3x^2$, then b is equal to
    \hfill{\brak{\text{Mar 2021}}}

	\begin{enumerate}
		\item $5$ 
		\item $6$
		\item $12$
		\item $10$
	\end{enumerate}

    %Question20
    \item Let $A = \cbrak{2, 3, 4, 5, \dots, 30}$ and '$\tilde{=}$' be an equivalence relation on $A \times A$, defined by $\brak{a, b} \tilde{=} \brak{c, d}$, if and only if $ad = bc$. Then the number of ordered pairs which satisfy this equivalence relation with ordered pair $\brak{4, 3}$ is equal to
    \hfill{\brak{\text{Mar 2021}}}

	\begin{enumerate}
		\item $7$ 
		\item $5$
		\item $6$
		\item $8$
	\end{enumerate}